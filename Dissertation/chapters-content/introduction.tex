\section{Motivation}

Breast cancer is one of the most common forms of cancer amongst women in the UK, with statistics indicating that 1 in 7 females will be diagnosed with breast cancer in their lifetime. Indeed, 55,200 new breast cancer cases are reported every year in the UK, of which an average of 11,400 lead to death (20\% mortality rate) \citep{BreastCancerResearchUK}. However, many other types of cancer types exist, ranging from lung and prostate cancers to colon and bladder cancers to name a few \citep{cokkinides2005american}.\\

The goal of this project is to design a deep learning pipeline that can learn how to detect cases of breast cancer in mammograms (X-ray pictures of breasts) by detecting the presence of tumours, whether they are benign (non-cancerous) or malignant (cancerous), and their location in the mammogram. The deep learning algorithm will learn the underlying patterns of a large dataset of mammograms to carry out the aforementioned tasks. The motivation behind this project is to allow a technique for early and accurate breast cancer detection to prevent unnecessary treatments in cases of false positives and to prevent late treatments due to false negatives. Ultimately, the target of this project is to combine it with deep learning algorithms developed across other projects supervised by Dr David Harris-Birtill (past and present). This will allow a general artificial intelligence system capable of detecting multiple forms of cancer with higher accuracies than radiologist diagnostics.\\

Parts of the work undertaken during this project will be conducted as a group comprised of two other members. The tasks in common involve the data cleaning and pre-processing, the results output and the implementation of a basic deep learning pipeline. The reasoning for these common tasks is to allow each group member to further explore deep learning techniques on their own using the common basic pipeline as a starting point and to compare final results at the end. Section X covers in more detail which tasks will be conducted personally and in a group.

%%%%%%%%%%%%%%%%%%%%%%%%%%%%%%%%%%%%%%%%%%%%%%%%%%%%%%%%%%%%%%%%%%%%%%%%%%%%%%%%%%

\section{Problem Description}
\label{sec:problem-description}

To complete.

%%%%%%%%%%%%%%%%%%%%%%%%%%%%%%%%%%%%%%%%%%%%%%%%%%%%%%%%%%%%%%%%%%%%%%%%%%%%%%%%%%

\section{Report Structure}

\tab \textbf{Introduction} \space 
Presents the subject's background through a problem description and the motivation to tackle this problem.\\

\textbf{Context Survey} \space
Explores the literature and background surrounding breast cancer detection techniques, starting from primitive systems, exploring early machine learning methods, and ending with the deep learning techniques nowadays.\\

\textbf{Ethics \& Datasets} \space
Considers the ethical issues taken into account for this project and describes the various datasets used.\\

\textbf{Design} \space
Details the general structure of the system, offering an analysis of different parts of the deep learning pipeline without going into technical code-related or mathematics-related details. For example, this includes the choice of input method, data pre-processing steps, output format and programming languages and libraries.\\

\textbf{Implementation} \space
Covers the specifics followed when implementing the deep learning pipeline, explaining the mathematical and software-related reasoning behind each decision (this will include equations and code snippets). Additionally, covers any testing done to validate that the system works as expected.\\

\textbf{Evaluation} \space
Compare the final results with the basic pipeline’s results developed in common, the final results achieved by the two other group members and the results from papers researching breast cancer detection identified in the context survey.\\

\textbf{Conclusions} \space
This sections will summarise the project as a whole, from the initial objective to the results obtained. Further discussions will be included to objectively assess what could have improved, as well as any potential future work.