\section{Motivation}

Breast cancer is one of the most common forms of cancer amongst women in the UK, with statistics indicating that 1 in 7 females will be diagnosed with breast cancer in their lifetime. Indeed, 55,200 new breast cancer cases are reported every year in the UK, of which an average of 11,400 lead to death (20\% mortality rate) \citep{BreastCancerResearchUK}. However, many other types of cancer types exist, ranging from lung and prostate cancers to colon and bladder cancers to name a few \citep{cokkinides2005american}.\\

The goal of this project is to design a deep learning pipeline that can learn how to detect cases of breast cancer in mammograms (X-ray pictures of breasts) by detecting the presence of tumours, whether they are benign (non-cancerous) or malignant (cancerous), and their location in the mammogram. The deep learning algorithm will learn the underlying patterns of a large dataset of mammograms to carry out the aforementioned tasks. The motivation behind this project is to allow a technique for early and accurate breast cancer detection to prevent unnecessary treatments in cases of false positives and to prevent late treatments due to false negatives. Ultimately, the target of this project is to combine it with deep learning algorithms developed across other projects supervised by Dr David Harris-Birtill (past and present). This will allow a general artificial intelligence system capable of detecting multiple forms of cancer with higher accuracies than radiologist diagnostics.\\

Parts of the work undertaken during this project will be conducted as a group comprised of two other members. The tasks in common involve the data cleaning and pre-processing, the results output and the implementation of a basic deep learning pipeline. The reasoning for these common tasks is to allow each group member to further explore deep learning techniques on their own using the common basic pipeline as a starting point and to compare final results at the end. Section~\ref{sec:introduction-objectives} covers in more detail which tasks will be conducted personally and in a group.

%%%%%%%%%%%%%%%%%%%%%%%%%%%%%%%%%%%%%%%%%%%%%%%%%%%%%%%%%%%%%%%%%%%%%%%%%%%%%%%%%%

\section{Problem Description}
\label{sec:problem-description}

To complete.

%%%%%%%%%%%%%%%%%%%%%%%%%%%%%%%%%%%%%%%%%%%%%%%%%%%%%%%%%%%%%%%%%%%%%%%%%%%%%%%%%%

\section{Objectives}
\label{sec:introduction-objectives}

The main objective of this project consists of implementing a deep learning pipeline that will be able to learn how to detect cases of breast cancer in mammograms. This objective is broken down into two steps:
\begin{itemize}
    \item \textbf{Group work}: a common deep learning pipeline will be initially implemented as part of a group with Ashay Patel and Shuen-Jen Chen over the course of a three-week period. The distribution of tasks between the group can be found in Appendix~\ref{ch:appendix-team-meeting-summaries}.
    \item \textbf{Individual work}: the aforementioned pipeline will then be individually extended by introducing techniques that differ from the ones found in existing literature.
\end{itemize}

To reach this goal, an extensive literature review has to first be written to cover the background of deep learning techniques applied to the field of cancer detection and to review existing results. These include identifying results achieved using different methods (e.g. traditional machine learning techniques) and reviewing state of the art deep learning techniques applied to the detection of breast cancer in mammograms. This step is primordial as it will guide the research towards the most promising areas, as well as govern the choice of techniques to implement and explore in further sections.\\

The final results achieved individually will finally be compared with the common deep learning pipeline designed in group, as well as the results gathered by the two other group members and the results found in papers that used similar approaches.

%%%%%%%%%%%%%%%%%%%%%%%%%%%%%%%%%%%%%%%%%%%%%%%%%%%%%%%%%%%%%%%%%%%%%%%%%%%%%%%%%%

\section{Report Structure}

\tab \textbf{Introduction} \space 
Presents an overview of the subject's background through the problem description and the motivation behind this project, followed by the objectives that project aims to achieve.\\

\textbf{Context Survey} \space
Explores the literature and background surrounding breast cancer detection techniques, starting from primitive cancer detection systems, followed by traditional machine learning methods, and ending with the deep learning techniques that have been used in recent times.\\

\textbf{Ethics \& Datasets} \space
Considers the ethical issues taken into account for this project and describes the datasets used.\\

\textbf{Design} \space
A high-level exploration of design considerations regarding the deep learning pipeline to implement and the software in general.\\

\textbf{Implementation} \space
Comprehensively covers the steps followed when implementing the deep learning pipeline, explaining the practical solutions to implement code. Additionally, covers any testing done to validate that the system works as expected.\\

\textbf{Evaluation} \space
A review of the different results to assess the efficiency of the different techniques used to train the and how it compares to other models, including the common pipeline, the other group member's results, and relevant results identified in the context survey.\\

\textbf{Conclusions} \space
A summary the accomplished project's objectives, its limitations, plans for future work, and a final reflection on the project as a whole.