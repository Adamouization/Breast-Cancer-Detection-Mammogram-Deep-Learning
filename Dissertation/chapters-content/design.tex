\section{Deep Learning Pipeline Design Analysis}

\subsection{Data pre-processing}

todo

\subsection{CNN Model}

todo

\subsection{Output}

Output metrics chosen: a combination of numerical metrics and visual metrics.

\begin{itemize}
    \item numerical metrics:
    \begin{itemize}
        \item overall accuracy
        \item precision
        \item recall
        \item F1 score
    \end{itemize}
    \item visual metrics:
    \begin{itemize}
        \item confusion matrix
        \item normalised confusion matrix
        \item Receiver Operating Characteristic (ROC) curve
    \end{itemize}
\end{itemize}


%%%%%%%%%%%%%%%%%%%%%%%%%%%%%%%%%%%%%%%%%%%%%%%%%%%%%%%%%%%%%%%%%%%%
%%%%%%%%%%%%%%%%%%%%%%%%%%%%%%%%%%%%%%%%%%%%%%%%%%%%%%%%%%%%%%%%%%%%
%%%%%%%%%%%%%%%%%%%%%%%%%%%%%%%%%%%%%%%%%%%%%%%%%%%%%%%%%%%%%%%%%%%%

\section{General Design Decisions}

\subsection{Programming language}

Python 3 with the usual ML suite available.
Tensorflow \& Keras over PyTorch. % more widely used in industry and older, so more guides online.

\subsection{Interface}

A Command-Line Interface (CLI) is selected, allow arguments and flags to be passed to execute different sections of the code. Arguments control the dataset to use, the CNN model to use and the mode to run in (training or testing). Flags control the verbose mode to print more statements in the terminal for debugging purposed.

%%%%%%%%%%%%%%%%%%%%%%%%%%%%%%%%%%%%%%%%%%%%%%%%%%%%%%%%%%%%%%%%%%%%
%%%%%%%%%%%%%%%%%%%%%%%%%%%%%%%%%%%%%%%%%%%%%%%%%%%%%%%%%%%%%%%%%%%%
%%%%%%%%%%%%%%%%%%%%%%%%%%%%%%%%%%%%%%%%%%%%%%%%%%%%%%%%%%%%%%%%%%%%

\clearpage
\section{Chosen Solution}

\begin{enumerate}
    \item \underline{\textbf{Programming Language}}: 
    \begin{enumerate}
        \item Python 3.7
        \item Third-party libraries:
        \begin{enumerate}
        	\item Keras
        	\item TensorFlow
            \item NumPy
            \item Pandas
            \item SciPy
            \item Matplotlib
        \end{enumerate}
    \end{enumerate}
\end{enumerate}