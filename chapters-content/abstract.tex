% https://www.scribbr.co.uk/thesis-dissertation/abstract/

%%TC:ignore

% Aims
The objective of this dissertation is to explore various deep learning techniques that can be used to implement a system which learns how to detect instances of breast cancer in mammograms. Nowadays, breast cancer claims 11,400 lives on average every year in the UK, making it one of the deadliest diseases. Mammography is the gold standard for detecting early signs of breast cancer, which can help cure the disease during its early stages. However, incorrect mammography diagnoses are common and may harm patients through unnecessary operations (or a lack of operations). Therefore, systems that can learn to detect breast cancer on their own could help reduce the number of incorrect interpretations and missed cases.\\

% Methods
Convolution Neural Networks (CNNs) are used as part of a deep learning pipeline initially developed in a group and further extended individually. A bag-of-tricks approach is followed to analyse the effects on performance and efficiency using diverse deep learning techniques such as different architectures (VGG19, ResNet50, InceptionV3, DenseNet121, MobileNetV2), class weights, input sizes, amounts of transfer learning, and types of mammograms.\\

% Results
Describe results\\

% Code
The code developed for this dissertation can be found online at the following URL: \url{https://github.com/Adamouization/Breast-Cancer-Detection-and-Segmentation}. The code developed as a group at the beginning of the dissertation can also be found online at the following URL: \url{https://doi.org/10.5281/zenodo.3975093} \citep{adam_jaamour_2020_3975093}.

\vspace{12mm}
\begin{center}

\textbf{Keywords}\\
\vspace{5mm}
Deep Learning; Convolution Neural Networks; Transfer Learning; Breast Cancer Detection; Mammogram Classification; CBIS-DDSM; mini-MIAS

\end{center}

%%TC:endignore