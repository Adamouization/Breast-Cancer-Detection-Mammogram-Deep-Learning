\section{Installation Instructions}

Start by cloning the GitHub repository (either the common or the individual code):

\begin{lstlisting}
cd ~/Projects
git clone https://github.com/Adamouization/Breast-Cancer-Detection-Code
\end{lstlisting}

Create a repository that will be used to install Tensorflow 2 with CUDA 10 for Python and activate the virtual environment for GPU usage:

\begin{lstlisting}
cd libraries/tf2
tar xvzf tensorflow2-cuda-10-1-e5bd53b3b5e6.tar.gz
sh build.sh
\end{lstlisting}

Activate the virtual environment:

\begin{lstlisting}
source /Breast-Cancer-Detection-Code/tf2/venv/bin/activate
\end{lstlisting}

``\textit{cd}'' into the ``\textit{src}'' directory and run the code below.

%%%%%%%%%%%%%%%%%%%%%%%%%%%%%%%%%%%%%%%%%%%%%%%%%%%%%%

\section{Individual Code Instructions}
\label{sec:appendix-individual-pipeline-instructions}

Run the code:

\begin{lstlisting}
main.py [-h] -d DATASET [-mt MAMMOGRAMTYPE] -m MODEL [-r RUNMODE] [-lr LEARNING_RATE] [-b BATCHSIZE] [-e1 MAX_EPOCH_FROZEN] [-e2 MAX_EPOCH_UNFROZEN] [-gs] [-roi] [-v]
\end{lstlisting}

where:
\begin{itemize}
    \item \textit{-h} is a  flag for help on how to run the code.
    \item \textit{DATASET} is the dataset to use. Must be either \textit{mini-MIAS}, \textit{mini-MIAS-binary} or \textit{CBIS-DDMS}.
    \item \textit{MAMMOGRAMTYPE} is the type of mammogram to use. Can be either \textit{calc}, \textit{mass} or \textit{all}.
    \item \textit{MODEL} is the model to use. Must be either \textit{VGG}, \textit{VGG-common}, \textit{Inception} or \textit{CNN}. Default value is \textit{VGG-common}.
    \item \textit{RUNMODE} is the learning rate used for the all the non-pre-trained ImageNet layers. Defaults to 1e-3. Must be a float.
    \item \textit{LEARNING\_RATE} is the optimiser's initial learning rate to use when training the model during the first phase (frozen layers). Defaults to \textit{0.001}. Must be a positive float.
    \item \textit{BATCHSIZE} is the batch size to use when training the model. Defaults to \textit{2}. Must be a positive integer.
    \item \textit{MAXEPOCHFROZEN} is the maximum number of epochs in the first training phrase (with frozen layers). Defaults to \textit{100}. Must be a positive integer.
    \item \textit{MAXEPOCHUNFROZEN} is the maximum number of epochs in the second training phrase (with unfrozen layers). Defaults to \textit{50}. Must be a positive integer.
    \item \textit{-roi} is a flag to use only cropped versions of the images around the ROI. Only usable with mini-MIAS dataset. Defaults to \textit{False}.
    \item \textit{-v} is a flag controlling verbose mode, which prints additional statements for debugging purposes. Defaults to \textit{False}.
\end{itemize}

%%%%%%%%%%%%%%%%%%%%%%%%%%%%%%%%%%%%%%%%%%%%%%%%%%%%%%

\section{Common Pipeline Code Instructions}
\label{sec:appendix-common-pipeline-instructions}

Run the code:

\begin{lstlisting}
python main.py [-h] -d DATASET -m MODEL [-r RUNMODE] [-i IMAGESIZE] [-v]
\end{lstlisting}

where:
\begin{itemize}
    \item \textit{-h} is a  flag for help on how to run the code.
    \item \textit{DATASET} is the dataset to use. Must be either \textit{mini-MIAS} or \textit{CBIS-DDMS}.
    \item \textit{MODEL} is the model to use. Must be either \textit{basic} or \textit{advanced}.
    \item \textit{RUNMODE} is the mode to run in (\textit{train} or \textit{test}). Default value is \textit{train}.
    \item \textit{IMAGESIZE} is the image size to feed into the CNN model (\textit{small} - 512x512px; or \textit{large} - 2048x2048px). Default value is \textit{small}.
    \item \textit{-v} is a flag controlling verbose mode, which prints additional statements for debugging purposes.
\end{itemize}

%%%%%%%%%%%%%%%%%%%%%%%%%%%%%%%%%%%%%%%%%%%%%%%%%%%%%%

\section{Dataset Installation Instructions}

\subsection{mini-MIAS dataset}

This example will use the mini-MIAS dataset\footnote{mini-MIAS dataset: \url{http://peipa.essex.ac.uk/info/mias.html}}. After cloning the project, travel to the \textit{data/mini-MIAS} directory (there should be 3 files in it).\\

Create \textit{images\_original} and \textit{images\_processed} directories in this directory: 

\begin{lstlisting}
cd data/mini-MIAS/
mkdir images_original
mkdir images_processed
\end{lstlisting}

Move to the \textit{images\_original} directory and download the raw un-processed images:

\begin{lstlisting}
cd images_original
wget http://peipa.essex.ac.uk/pix/mias/all-mias.tar.gz
\end{lstlisting}

Unzip the dataset then delete all non-image files:

\begin{lstlisting}
tar xvzf all-mias.tar.gz
rm -rf *.txt 
rm -rf README 
\end{lstlisting}

Move back up one level and move to the \textit{images\_processed} directory. Create 3 new directories there (\textit{benign\_cases}, \textit{malignant\_cases} and \textit{normal\_cases}):

\begin{lstlisting}
cd ../images_processed
mkdir benign_cases
mkdir malignant_cases
mkdir normal_cases
\end{lstlisting}

Now run the python script for processing the dataset and render it usable with Tensorflow and Keras:

\begin{lstlisting}
python3 ../../../src/data_manipulations/mini-MIAS-initial-pre-processing.py
\end{lstlisting}

\subsection{CBIS-DDSM dataset}

These datasets are very large (exceeding 160GB) and more complex than the mini-MIAS dataset to use. They were downloaded by the University of St Andrews School of Computer Science computing officers onto \textit{BigTMP}, a 15TB filesystem that is mounted on the Centos 7 computer lab clients with NVIDIA GPUs usually used for storing large working data sets. Therefore, the download process of these datasets will not be covered in these instructions.\\

The generated CSV files to use these datasets can be found in the \textit{/data/CBIS-DDSM} directory, but the mammograms will have to be downloaded separately directly from the source. The CBIS-DDSM dataset can be downloaded here: \url{https://wiki.cancerimagingarchive.net/display/Public/CBIS-DDSM#5e40bd1f79d64f04b40cac57ceca9272}.
